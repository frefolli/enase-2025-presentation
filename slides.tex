\documentclass[table]{beamer}
\usepackage{graphicx}
\usepackage{listings}
\usepackage{xcolor}

\lstset{basicstyle=\footnotesize\ttfamily,
        keywordstyle=\footnotesize\color{blue}\ttfamily}

\lstdefinestyle{tssg}{
  %belowcaptionskip=1\baselineskip,
  %breaklines=true,
  %frame=L,
  %xleftmargin=\parindent,
  showstringspaces=false,
  keywordstyle=\bfseries\color{green!40!black},
  morekeywords={let, node, attr, edge, "=", "->"},
  identifierstyle=\color{blue},
  stringstyle=\color{orange},
}

\lstdefinestyle{yaml}{
  keywordstyle=\bfseries\color{green!40!black},
  morekeywords={nodes, edges, filepaths, name, kind, source, sink},
  % identifierstyle=\color{cyan},
}
 
\usetheme{Madrid}
%\usecolortheme{default}
\usecolortheme{beaver}
\date[4 April 2025]{}

\title{Lessons learned from implementing a language-agnostic dependency graph parser}
%\subtitle{ENASE 2025}
\author[F.~Refolli \and ~D.~Sas \and F.~Arcelli~Fontana]{\textbf{Francesco~Refolli} \and ~Darius~Sas \and Francesca~Arcelli~Fontana}
%\logo{\includegraphics[height=1cm]{logo_unimib.pdf}}
\titlegraphic{\includegraphics[height=3cm]{logo_unimib.pdf}\hspace*{4.75cm}~\includegraphics[height=3cm]{enase_logo.jpg}}

\newcommand{\putimage}[2] {
  \begin{figure}[H]
    \centering
    \includegraphics[width=#2\linewidth]{#1}
	\end{figure}
}

\newcommand{\putimagecouple}[4] {
  \begin{figure}[!htb]
    \centering
    \begin{minipage}{0.45\linewidth}
      \centering
      \includegraphics[width=\linewidth]{#1}
      \caption{#2}
    \end{minipage}
    \hspace{0.25cm}
    \begin{minipage}{0.45\linewidth}
      \centering
      \includegraphics[width=\linewidth]{#3}
      \caption{#4}
    \end{minipage}
  \end{figure}
}

\begin{document}

\frame{\titlepage}

\begin{frame}
\frametitle{Outline}
\tableofcontents
\end{frame}

\section{Introduction}
\begin{frame}
  \centering
  \huge Introduction
\end{frame}

\begin{frame}
  \frametitle{How do software analysis tools work?}
  \putimage{images/d2/software-analysis-tools.png}{0.99}
\end{frame}

\begin{frame}
  \frametitle{A tedious detail ...}
  Not every project is written in X language!
  \putimage{images/d2/the-tedious-detail-1.png}{0.99}
\end{frame}

\begin{frame}
  \frametitle{A tedious detail ...}
  Some projects are even blends of languages!
  \putimage{images/d2/the-tedious-detail-2.png}{0.99}
\end{frame}

\begin{frame}
  \frametitle{A "naive" solution}
  \putimage{images/d2/naive-solution.png}{0.99}
\end{frame}

\begin{frame}
  \frametitle{Other techniques}
  \putimage{images/d2/techniques.png}{0.99}
\end{frame}

\begin{frame}
  \frametitle{What we want}
  A few key principles:
  \begin{itemize}
    \item An efficient intermediate representation
    \item Avoid source code / syntax tree translations
    \item Avoid language-specific dependencies
    \item A Maintainable, extensible and generic approach
    \item Easily add support for a new language
    \item Build it right and built it fast
  \end{itemize}
\end{frame}

\section{The tools}
\begin{frame}
  \centering
  \huge The tools
\end{frame}

\begin{frame}[fragile]
  \frametitle{Tree Sitter}
  \begin{columns}
    \begin{column}{0.50\textwidth}
      \begin{itemize}
        \item Is a parser generator
				\item Parses code into a Concrete Syntax Tree (CST)
      \end{itemize}
    \end{column}
    \begin{column}{0.50\textwidth}
      \lstinputlisting[language=Java]{listings/technology-example.java}
    \end{column}
  \end{columns}
  \putimage{images/plain/tree-sitter-example.png}{0.99}
\end{frame}

\begin{frame}[fragile]
  \frametitle{Stack Graphs}
  \begin{columns}
    \begin{column}{0.50\textwidth}
      \begin{itemize}
        \item Composable graph
        \item Represents identifiers and scopes of source code
        \item Enables language agnostic reference resolution
      \end{itemize}
    \end{column}
    \begin{column}{0.50\textwidth}
      \lstinputlisting[language=Java]{listings/technology-example.java}
    \end{column}
  \end{columns}
  \putimage{images/d2/stack-graph-example.png}{0.99}
\end{frame}

\begin{frame}[fragile]
  \frametitle{Tree Sitter Stack Graphs}
  \begin{itemize}
    \item A "TSSG" Grammar is composed of pairs
    \item Each pair matches a section of the CST with some procedural code
    \item Definition of nodes, edges and labels
  \end{itemize}
  \lstinputlisting[style=tssg]{listings/tssg-pair-example.txt}
\end{frame}

\section{Building the dependency graph}
\begin{frame}
  \centering
  \huge Building the dependency graph
\end{frame}

\begin{frame}
  \frametitle{How is the Stack Graph built?}
  \begin{itemize}
    \item Definition identifiers get a \textit{pop} node
    \item Reference identifiers get a \textit{push} node
    \item \textit{Scope} nodes appear at various levels to allow navigation
    \item Chaining nodes with edges allow to resolve complex references. Examples:
      \begin{itemize}
        \item "$java \leftarrow util \leftarrow Scanner$" defines a sequential search for its identifies.
        \item "$point \leftarrow x$" employs type resolution and field access.
      \end{itemize}
    \item Try to avoid cycles to reduce reference resolution time
    \item Apply \textit{refkind/defkind} labels to capture fine-grain relationships and component types
  \end{itemize}
\end{frame}

\begin{frame}
  \frametitle{How are references resolved?}
  \putimage{images/mmd/reference-resolution-algorithm.png}{0.60}
\end{frame}

\begin{frame}
  \frametitle{How is the dependency graph built?}
  \begin{itemize}
    \item Depth-first visit of the Stack Graph from root
    \item Add each definition node to the component graph
    \item Keep track of structural dependencies
    \item Decorate the graph with references resolved previously
    \item Serialization with JSON, GraphML ... etc
  \end{itemize}
\end{frame}

\begin{frame}[fragile]
  \frametitle{Another basic example / 1}
  \lstinputlisting[language=Java]{listings/dependency-graph-example.java}
\end{frame}

\begin{frame}
  \frametitle{Another basic example / 2}
  \putimage{images/d2/dependency-graph-example.png}{0.99}
\end{frame}

\section{Evaluation}
\begin{frame}
  \centering
  \huge Evaluation
\end{frame}

\begin{frame}[fragile]
  \frametitle{How is the prototype evaluated?}
  Describe assertion on files, nodes and edges
  \begin{columns}
    \begin{column}{0.64\textwidth}
      \lstinputlisting[style=yaml]{listings/policy-test-example.yml}
    \end{column}
    \begin{column}{0.34\textwidth}
      \begin{itemize}
        \item node types: class, function, enum ...
        \item edge types: definedBy, includes, usesType, calls ...
      \end{itemize}
    \end{column}
  \end{columns}
\end{frame}

% TODO: aggiungere una slide su Arcan con il logo di Arcan e il diagramma dei componenti che hai messo nell'articolo

\begin{frame}
  \frametitle{The tool Arcan}
  \putimage{images/plain/arcanDepgraph.png}{0.70}
\end{frame}

\begin{frame}
  \frametitle{The Pruijt et al's benchmark}
	% TODO: aggiungere la fonte del Benchmark
  \putimage{images/py/prototype-arcan-pruijt.png}{0.80}

	\footnote{The accuracy of dependency analysis in static architecture compliance checking, Pruijt et Al, Softw. Pract. Exp. 2017}
\end{frame}

\begin{frame}
  \frametitle{Running on real-world scenarios}
    \center
    \begin{tabular}{|l l c r|}
      \hline
      \textbf{Project} & \textbf{Language} & \textbf{Version} & \textbf{Size (in LOC)} \\
      \hline
      JUnit4 & Java & 4 & 30K \\
      \hline
      JUnit5 & Java & 5 & 100K \\
      \hline
      ANTLR & Java & 4 & 180K \\
      \hline
      Fastjson & Java & 1 & 50K \\
      \hline
    \end{tabular}
    \center
    \begin{tabular}{|l|l|l|l|l|}
        \hline
        \textbf{Project} & \textbf{Tool} & \textbf{Min} & \textbf{Max} & \textbf{Mean execution time} \\ \hline
        \hline
        \rowcolor[HTML]{EED49F} 
        JUnit4   & prototype & 65,82  & 69,07  & 65,88     \\ \hline
        \rowcolor[HTML]{A6DA95} 
        JUnit4   & Arcan     & 13,850 & 17,079 & 14,611    \\ \hline
        \rowcolor[HTML]{EED49F} 
        JUnit5   & prototype & 132,87 & 134,75 & 134,44    \\ \hline
        \rowcolor[HTML]{A6DA95} 
        JUnit5   & Arcan     & 42,542 & 47,613 & 44,186    \\ \hline
        \rowcolor[HTML]{EED49F} 
        ANTLR    & prototype & 171,65 & 175,49 & 172,64    \\ \hline
        \rowcolor[HTML]{A6DA95} 
        ANTLR    & Arcan     & 19,222 & 20,140 & 19,691    \\ \hline
        \rowcolor[HTML]{ED8796} 
        Fastjson & prototype & N/A    & N/A    & N/A       \\ \hline
        \rowcolor[HTML]{A6DA95} 
        Fastjson & Arcan     & 66,932 & 71,094 & 69,071    \\ \hline
    \end{tabular}
\end{frame}

\section{Lessons learned}
\begin{frame}
  \centering
  \huge Lessons learned
\end{frame}

\begin{frame}
  \frametitle{Advantages}
  \centering
  \begin{itemize}
    \item Reference resolution and dependency inference are not tied to third party dependencies
    \item Achieves decent precision (comparable with the tool Arcan)
    \item Approach exportable to other languages (ex: Python, C++)
    \item Writing a TSSG grammar is easier than implementing a language frontend \\
    (but it can be harder than writing bindings to those)
  \end{itemize}
\end{frame}

\begin{frame}
  \frametitle{Shortcomings}
  \centering
  \begin{itemize}
    \item Poor performance
    \item Generally large Stack Graphs
    \item Lack of support for dependencies to external libraries
    \item Limited resolution for complex references
    \item The TSSG DSL is fairly limited (can lead to labeling conflicts)
    \item TSSG Grammars are long and not easy to read
  \end{itemize}
\end{frame}

\section{Conclusion}
\begin{frame}
  \centering
  \huge Conclusion
\end{frame}

\begin{frame}
  \frametitle{Conclusions}
  Summarizing:
  \begin{itemize}
    \item The presented approach works for small-to-medium projects
    \item Further work is needed to scale project sizes
  \end{itemize}

  \vspace{0.9cm}

  Future work:
  \begin{itemize}
    \item Experiment with automatic graph construction
    \item Build abstractions upon Tree Sitter CSTs
  \end{itemize}
\end{frame}

\begin{frame}
  \centering
  \huge Thank you for the attention
\end{frame}

\end{document}
